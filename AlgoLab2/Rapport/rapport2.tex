\documentclass[10pt,a4paper]{article}
\usepackage{geometry}
\usepackage[french]{babel}
\usepackage[utf8]{inputenc}
\usepackage[T1]{fontenc}
\usepackage{lmodern} \normalfont
\DeclareFontShape{T1}{lmr}{bx}{sc}{<-> ssub * cmr/bx/sc}{}
\usepackage{textcomp}
\usepackage{datetime}
\usepackage{amsmath}
\usepackage{amssymb}
\usepackage{graphicx}
\usepackage{wrapfig}
\usepackage{subcaption}
\usepackage{tocloft}
\usepackage{fixltx2e}
\usepackage{color}
\usepackage[colorlinks=true,
			linkcolor=blue,
			bookmarksnumbered=true,
			pdftitle={Rapport INF4705},
			pdfauthor={Gwenegan Hudin},
			pdfborder={0 0 0},
			pdfsubject={Stage 3INFO}]{hyperref}

% Custom commands
\newcommand{\HRule}{\rule{\linewidth}{0.5mm}}
\newcommand{\Section}[1]{\section*{#1} \addcontentsline{toc}{section}{#1} \setcounter{subsection}{0}}
%\renewcommand*{\theHsection}{chY.\the\value{section}}
\renewcommand{\thesection}{\Roman{section}.}
\renewcommand{\thesubsection}{\arabic{subsection}.}
\renewcommand{\thesubsubsection}{\alph{subsubsection}.}
\renewcommand{\cftsecnumwidth}{2em}
\renewcommand{\cftsubsecnumwidth}{2em}
\renewcommand{\cftsubsubsecnumwidth}{2em}
\addto\captionsfrench{
	\renewcommand{\cfttoctitlefont}{\Large}
	\renewcommand{\contentsname}{\centering \textsc{Table des Matières}\\[0.5cm]}
}

\renewcommand{\baselinestretch}{1.15}

\begin{document}

\begin{titlepage}
	\begin{center}
		\begin{figure}
        \begin{subfigure}[c]{0.2\textwidth}
        		\centering
                \includegraphics[width=0.6\textwidth]{images/logo-polymtl}
        \end{subfigure}
		\end{figure}
		
		
		\vspace{30pt}
		\textsc{\huge Génie Informatique}\\
		\textsc{\LARGE Rapport de Travaux Pratiques}\\		
		\vfill
		
		% Title
		\HRule \\[0.7cm]
		{\Huge \bfseries INF4705 : Lab 2}\\[0.4cm]
		{\Large Algorithmes voraces et dynamiques : Application au tri topologique d'un graphe}\\[0.2cm]
		\HRule\\[1cm]
		
		\vfill
		
		% Author
		\begin{minipage}{0.49\textwidth}
			\begin{flushleft} \LARGE
				\textbf{Auteur}\\
				Gwenegan \textsc{Hudin}\\ 1756642\\[0.5cm]
			\end{flushleft}
		\end{minipage}
		\begin{minipage}{0.49\textwidth}
			\begin{flushright} \LARGE
				\textbf{Rendu}\\
				14 Novembre 2014\\ À Polytechnique Montréal\\[0.5cm]
			\end{flushright}
		\end{minipage}
	\end{center}
\end{titlepage}

\newpage

\hfill

\newpage

\tableofcontents

\newpage

\section{Introduction}

Lors de cette expérience, nous allons couvrir le problème de tri topologique de graphes orientés acycliques selon trois approches : vorace, retour arrière et dynamique. Plus que les tris en eux-mêmes, nous nous intéresserons au nombre de permutations de tri détectées par chaque algorithme, et leurs vitesses respectives. Ces permutations seront nommées \og extensions linéaires \fg d'un graphe.

Au début de l'expérience, nous n'avons pas de suppositions de précédence d'une méthode sur une autre, et nous n'avons donc pas d'hypothèse précise à vérifier, contrairement au rapport précédent. Nous pouvons simplement supposer que l'algorithme vorace, dont l'utilisation afin de déterminer le nombre d'extensions linéaires repose sur une approximation par formule mathématique, sera moins précis que les deux autres méthodes.

L'on peut aussi supposer que l'algorithme de programmation dynamique, reposant sur une structure de données à $N$ dimensions, sera bien plus coûteux et difficile à implémenter.

Ce rapport a pour but de présenter les différents résultats obtenus avec chaque implémentation, appliquer des analyses asympotiques et hybrides, et de discuter des méthodes utilisées.

\section{Revue de la théorie}

Dans ce rapport, tous les graphes évoqués seront considérés orientés et acycliques, et définis formellement comme suit : $ G = (S,A) $, avec $S$ ensemble de sommets $\{1, \ldots, n\}$ et $A$ ensemble d'arcs tels que $ A \subset S \times S $.

On définit un tri topologique de $G$ comme étant une permutation $ \sigma $ de $ S $ telle que $ (i,j) \in A \longrightarrow \sigma(i) < \sigma(j) $.

\subsection{Fonctionnement de l'algorithme vorace}

\subsection{Fonctionnement de l'algorithme retour arrière}

\subsection{Fonctionnement de l'algorithme dynamique}

\subsection{Étude de complexité}


\section{Protocole expérimental}

\subsection{Environnement de développement}

Les tests ont été effectués sur un système ArchLinux 64bits, kernel 3.16.3-1-ARCH en environnement Gnome3, sur une machine disposant de 8Go de mémoire vive et d'un processeur i7-2630QM cadencé à 2.00GHz. L'ordinateur portable a reposé sur une station ventilée pendant toute la durée des calculs.

Le programme a été écrit en C++ 11 et compilé avec GCC 4.9.1.

\subsection{Déroulement de l'expérience}

\section{Résultats}

\section{Analyse}

\subsection{Analyse asymptotique}

\subsubsection{Algorithme vorace}
\subsubsection{Algorithme retour arrière}
\subsubsection{Algorithme dynamique}

\subsection{Analyse hybride}

\subsection{Discussion}

\subsubsection{Qualité de réponse}

\subsubsection{Consommation de ressources}

\subsubsection{Implémentation}

\section{Conclusion}

\section{Bibliographie}

Aucune portion de code de l'implémentation ou de la représentation des résultats n'a été copié. Cependant, diverses inspirations ont permis la réalisation de cette expérience.

\begin{itemize}
	\item \href{http://msdn.microsoft.com/en-us/library/hh873134.aspx}{Microsoft Developer Network, Walkthrough: Matrix Multiplication}
	\item \href{http://www.cplusplus.com/reference/vector/vector/}{C++ Reference, Vector class}
	\item \href{http://www.cplusplus.com/doc/tutorial/arrays/}{C++ Reference, Array class}
	\item \href{http://www.cplusplus.com/reference/tuple/tuple/}{C++ Reference, Tuple class}
	\item \href{http://stackoverflow.com/questions/7868936/c-read-file-line-by-line}{Stack Overflow, Réponse de Kerrek SB à la question "Read file line by line"}
	\item Notes de cours "Algorithmes Diviser-pour-Regner", Gilles Pesant
\end{itemize}

\end{document}
